\input{./../header/kib.tex}
% Вопрос к аудитории 
\newcommand{\вопрос}[1]{\textbf{\textcolor{red}{#1}}}
\newcommand{\внекурса}[1]{\textbf{\textcolor{violet}{#1}}}  

% TODO ! напишите здесь название вашей лекции
\title{Лекция 1. Комбинаторика часть 1.}
% TODO ! замените на дату проведения этой лекции. Например \date{14 апреля 2019}
\date{<TODO дата>}
% \logo{\href{https://t.me/kibinfo}{\includegraphics[width=.05\textwidth]{./../pic/kib_logo.png}}}
\author{Слипенчук Павел Владимирович}
\institute{\centering \includegraphics[width=.2\textwidth]{./../pic/kib_logo.png} \\ Москва,\\ \href{https://t.me/kibinfo}{\textbf{КИБ}} }
% \titlegraphic{\href{https://t.me/kibinfo}{\includegraphics[width=.05\textwidth]{./../pic/kib_logo.png}}}

% TODO ! замените https://github.com/kib-courses/latex_templates на ссылку ВАШЕГО спецкурса!
\titlegraphic{\small \href{https://github.com/kib-courses/dsis-math-base}{Базовая математическая подготовка для Data Science}}

\begin{document}
  \maketitle
    
  \begin{frame}{План лекции}\label{frame:plan}
  	% TODO ! добавте в план все ваши секции, кроме "Вопросы для самопроверки", "Домашнее задание" и "Список материалов"
    \begin{enumerate}
	\item \nameref{section:why_combinatorics}
	\item \nameref{section:main_combinatorics_sxems}
	% \item \nameref{section:tkiz}
	% \item \nameref{section:another}
	\end{enumerate}
 \end{frame}
    
\section{Зачем нужна комбинаторика в DS}\label{section:why_combinatorics}
\begin{frame}
\termdef{Комбинаторный анализ (Комбинаторика)} — раздел математики, посвящённый 
решению задач выбора и расположения элементов некоторого, обычно конечного, множества
в соответствии с установленными правилами (схемами). 

Каждое такое правило определяет способ построения из элементов исходного множества некоторой конструкции, 
называемой \termdef{комбинаторной конфигурацией}. 
\end{frame}


\begin{frame}{Примеры задач комбинаторики}

\textbf{Простая задача.}
В школе танго 8 парней 
и 12 девушек.
Сколько существует всевозможных пар?

\begin{equation*}
8 \cdot 12 = 96
\end{equation*}

\textbf{Сложная задача.}
Сколько потребуется бит данных, 
чтобы сохранить всевозможные первые 20 ходов в шахматах?
\end{frame}


\begin{frame}{<<Задача о зёрнах на шахматной доске>>}
	
	\textbf{Когда очень много кажется малым...}
	
	Согласно индийской легенде, 
	брахман Сисса придумал шахматы (чатурангу).
	Правителю так понравилась игра, что он предложил Сиссе 
	выбрать себе награду.
	
	Хитрый Сисса попросил у правителя 
	на первую клетку положить одно зёрнышко пшеницы,
	на вторую два, 
	на третью четыре,
	на пятую восемь и так далее.
	
	Правитель, не разбиравшийся в комбинаторике,
	быстро согласился, даже несколько 
	обидевшись на столь "невысокую" цену.
	
	Однако неделю спустя 
	казначей доложил правителю что расплатиться невозможно,
	"разве что осушить моря и океаны и засеять всё пшеницей".
	
\end{frame}

\begin{frame}{<<Задача о зёрнах на шахматной доске>>}

\begin{equation*}
C = 1 + 2 + 4 + 8 + 16 + ... + 2^63 = \sum_{i=0}^{i=63} 2^i = 2^{64} - 1
=dfdf
\end{equation*}

Много или мало $2^{64}$ зёрен пшеницы ?

В 2021 году было произведено в мире 770 млн.тонн пшеницы.
Вес одного зерна примерно $50$ милиграмм.

Вопрос: сколько тысячелетий должно пройти, чтобы Сисса получил 
свою оплату?

\end{frame}

\begin{frame}
Задач, аналогичной классической 
<<Задаче о зёрнах...>>
в сфере Data Science чрезвычайно много.

Одна из целей комбинаторики:
понимать что такое МНОГО, ОЧЕНЬ МНОГО.

Например:
1. <<Давайте мониторить все видео на youtube с целью проверки на стеганографию!>>
2. <<ChatGPT заменит всех в call-центре!>>
 	
\end{frame}


\begin{frame}
	Road Map Data Science как мат.дисциплины \(кратко\)
	\begin{center}
		\begin{figure}
			\begin{tikzpicture}[node distance=1.5cm,auto]
\tikzstyle{recfill}=[draw, fill=blue!10, minimum size=2em]
\tikzstyle{rect}=[draw,minimum size=2em]
\tikzstyle{inv}=[draw,minimum size=2em]
% \tikzstyle{init} = [pin edge={to-,thin,black}]
\node (c) [recfill] {Комбинаторика};
\node (s) [below of=c, rect] {Мат.стат.};
\node (p) [left of=s, node distance=3.5cm, rect] {Теор.вер.};
\node (a) [right of=s, node distance=3.5cm, rect] {\small Алгоритмы и с.д.};
\node (m) [below of=p,rect] {Мат.ан.};
\node (l) [below of=s, rect] {Лин.ал.};
\node (inv) [below of=a] {};
\node (ml) [below of=inv, rect] {\small Machine Learning};

\path[->] (c) edge node {} (p);
\path[->] (c) edge node {} (a);
\path[->] (p) edge node {} (s);
\path[->] (m) edge node {} (s);
\path[->] (s) edge node {} (a);
\path[->] (m) edge node {} (a);
\path[->] (m) edge node {} (p);
\path[->] (m) edge node {} (l);
\path[->] (l) edge node {} (ml);
\path[->] (s) edge node {} (ml);
\path[->] (a) edge node {} (ml);

\node (sa) [below of=l, rect] {\small Предметная область};
\node (ca) [below of=m, rect] {\small Computer Science};

\node (ds) [below of=sa, rect] {\small Data Science};

\path[->] (ml) edge node {} (ds);
\path[->] (sa) edge node {} (ds);
\path[->] (ca) edge node {} (ds);

% \node [int, pin={[init]above:$v_0$}] (a) {$\frac{1}{s}$};
% \node (b) [left of=a,node distance=2cm, coordinate] {a};
% \node [int, pin={[init]above:$p_0$}] (c) [right of=a] {$\frac{1}{s}$};
% \node [coordinate] (end) [right of=c, node distance=2cm]{};
% \path[->] (b) edge node {$a$} (a);
% \path[->] (a) edge node {$v$} (c);
% \draw[->] (c) edge node {$p$} (end) ;
\end{tikzpicture}


			\caption{Место комбинаторики в DS}
		\end{figure}	
	\end{center}
	
\end{frame}

\section{Основные комбинаторные схемы}\label{section:main_combinatorics_sxems}

\begin{frame}{Комбинаторные принципы}
\termdef{Правило суммы. (Правило <<ИЛИ>>)}
Если элемент $A$ можно выбрать $a$ способами,
а элемент $B$ можно выбрать $b$ способами,
то выбрать $A$ или $B$ 
можно $a+b$ способами.

\termdef{Правило умножения. (Правило <<И>>)}
Если элемент $A$ можно выбрать $a$ способами,
а элемент $B$ можно выбрать $b$ способами,
то пару $(A, B)$ 
можно выбрать $a \cdot b$ способами.

\end{frame}

\begin{frame}{Комбинаторные принципы}

\termdef{Принцип включения-исключения}



\begin{equation}
\left| A \cup B \right| = \left|  A \right| + \left|  B \right| - \left| A \cap B \right|
\end{equation}


\end{frame}

\begin{frame}{Размещение (выборка без возвращения)}
Нужно из
$n$
различных элементов 
выбрать \textbf{упорядоченный}
ряд из $k$
элементов. 
(При этом, очевидно, что $k <=n$,
иначе задача не имеет решения.)

Сколько существует таких способов?   

Примеры:
1. TODO

\begin{equation}
A_n^k = n \cdot (n-1) \cdot (n-2) \cdot ... \cdot (n-k-1) = \frac{n!} {\left(n-k\right)!}
\end{equation}    

\вопрос{Как вывести эту формулу?}

\end{frame}

\begin{frame}
TODO 
\begin{equation*}
A_n^k = n \cdot (n-1) \cdot (n-2) \cdot ... \cdot (n-k-1)
\end{equation*}  
\end{frame}

\begin{frame}{Перестановка}
\termdef{Перестановка} -- 
частный случай \term{размещения}, 
когда $n=k$ 
\end{frame}


\section{Вопросы для самопроверки}

% \begin{frame}
% TODO
% \end{frame}

\section{Список материалов}

% \begin{frame}
% TODO
% \end{frame}

  
\end{document}