\input{./../header/kib.tex}
% Вопрос к аудитории 
\newcommand{\вопрос}[1]{\textbf{\textcolor{red}{#1}}}
\newcommand{\внекурса}[1]{\textbf{\textcolor{violet}{#1}}}  

% TODO ! напишите здесь название вашей лекции
\title{Лекция 4. Алгоритмы и структуры данных. Часть 1}
% TODO ! замените на дату проведения этой лекции. Например \date{14 апреля 2019}
\date{15.12.2023}
% \logo{\href{https://t.me/kibinfo}{\includegraphics[width=.05\textwidth]{./../pic/kib_logo.png}}}
\author{Слипенчук Павел Владимирович}
\institute{\centering \includegraphics[width=.2\textwidth]{./../pic/kib_logo.png} \\ Москва,\\ \href{https://t.me/kibinfo}{\textbf{КИБ}} }
% \titlegraphic{\href{https://t.me/kibinfo}{\includegraphics[width=.05\textwidth]{./../pic/kib_logo.png}}}

% TODO ! замените https://github.com/kib-courses/latex_templates на ссылку ВАШЕГО спецкурса!
\titlegraphic{\small \href{https://github.com/kib-courses/dsis-math-base}{Базовая математическая подготовка для Data Science}}

\begin{document}
  \maketitle
    


\begin{frame}{План лекции}\label{frame:plan}
	% TODO ! добавте в план все ваши секции, кроме "Вопросы для самопроверки", "Домашнее задание" и "Список материалов"
	\begin{enumerate}
		
		\item \nameref{section:introduction}
		\item \nameref{section:base_alg}
		\item \nameref{section:examples_1}
		\item \nameref{section:base_principles}
		\item \nameref{sections:base_problems}
		
	\end{enumerate}
\end{frame}

\section{Вводная. Список литературы.}\label{section:introduction}

\begin{frame}{Самая важная тема сегодняшней лекции...}
	\вопрос{А зачем вообще специалисту DSIS нужно глубоко знать алгоритмы и структуры данных? Все же есть в framework-ах!}
\end{frame}


\begin{frame}
\footnotesize
Алгоритмы и структуры данных -- несложный, но длинный курс в 1-2 семестра. 
За две лекции прочитать его невозможно. 

Цели двух лекций:
\begin{enumerate}
	\item ~<<Галопом по Европам>> -- рассказать о направлениях в этой дисциплине
	\item Методология АиСД -- научить учится.
\end{enumerate}

Методички (освоить в новогодние праздники):
\begin{enumerate}
	\item Фофанов О.Б. <<Алгоритмы и структуры данных. Учебное пособие>>, 123 с., Издательство ТПУ, 2014
	\item Бабичев С.Л. <<Лекции по алгоритмам и структурам данных>>, 394 с., самиздат, 2022
\end{enumerate}

\begin{block}{Замечание}
Первая работа очень простая и позволит быстро "въехать" в тему. 
Вторая -- методичка из Техносферы (МГУ) -- подготовка к "серьёзной книге" (см.след.слайд)
+ Обязательно прочитайте введение в кажой из них!
\end{block}

\end{frame}


\begin{frame}{<<Серьезная книга>>}

%TODO таблица -- картинка слева, текст справа

\includegraphics[width=0.4\textwidth]{./../pic/Kormen_3_edition_book_img.png}
<<Алгоритмы. Построение и анализ>>
Томас Кормен
Чарльз Лайзерсон
Рональд Ривест
Клиффорд Штайн.

\begin{block}{Замечание}
	Обязательно купите в бумаге 
	и медленно-медленно читайте...
	Возможно чтение займёт около года -- не пугайтесь. Оно того стоит!
\end{block}

\end{frame}



\section{Базовые понятия}\label{section:base_alg}

\begin{frame}{Алгоритм}
Алгоритм

Свойства алгоритма	
\end{frame}

\begin{frame}{Элементарные операции}
	
\end{frame}

\begin{frame}{Цена деления}
Деление -- сложная и дорогая вычислительная операция.

\begin{block}{Замечание}
Но не всегда так. 
Есть "табличный метод",
есть ПЛИС, видеокарты и т.д.

В которых \textbf{частные} случаи деления происходят быстро.
\end{block}

\end{frame}

\begin{frame}{Табличные и полутабличные методы}
	Табличный метод -- мы не вычисляем, а просто создаём большую-пребольшую таблицу (сложения, умножения, деления, возведения в степень и т.д.)
	
	\вопрос{Каковы преимущества и недостатки табличного метода? Когда невозможно использовать табличный методо?}
	
	Полутабличный метод -- мы частично используем таблицу, частично вычисляем.
	
	\вопрос{Можете привести пример? (см пример на слайде №\ref{frame:example_table_method})}.
\end{frame}



\section{Некотороые примеры}\label{section:examples_1}

\begin{frame}{Простой полутабличный алгоритм деления}\label{frame:example_table_method}
	Есть некое $N$, не слишком большое, чтобы таблицу $O(N)$ уместить в памяти.
	
	Есть некое плечо $l>>0$. Рассматриваем числа меньшие $l \cdot N$. 
	Можно построить упорядоченный вектор(список) этих чисел:
	\begin{equation}\label{eq:example_table_method_vector_def}
	(B_0=0, B_1 = l, B_2 = 2 \cdot l, ... B_i = i \cdot l, ..., B_N = N \cdot l )
	\end{equation} 
	
	Тогда любой $B_i$ можно вычислить по формуле:
	\begin{equation}\label{eq:example_table_method_B_i}
	\frac{B_i}{c} = \frac{i \cdot l}{c} = i \cdot \frac{l}{c}
	\end{equation}
	
	
 
 	\textbf{Задача.} Число $A < l \cdot N$ нужно разделить на $c<l$.
 	\вопрос{Как решить эту задачу с помощью вектора \eqref{eq:example_table_method_vector_def} и таблицы \eqref{eq:example_table_method_B_i}?}
	
	...
\end{frame}
\begin{frame}
 	...
 	
 	Решение. Можно заметить, что 
 	\begin{equation}
 	\exists i<l, \exists v<l: A = B_i+v
 	\end{equation}
 	Тогда верно:
 	\begin{equation}
 	\frac{A}{c} = \frac{B_i+v}{c} = \frac{B_i}{c} + \frac{v}{c} = i \cdot \frac{l}{c} + \frac{v}{c}
 	\end{equation}
 	
 	C помощью алгоритма поиска можно найти $B_i$ такой что: $B_i <= A < B_{i+1}$ % TODO знак <= на нормальный.
 	
 	Операция умножения дешёвая, операции деления $l$ и $v$ тоже не сильно догорие, 
 	так как $v<l$
 	
 	\дз{Посчитайте сложность этого алгоритма.}
 	
 
		
	
\end{frame}

\section{Базовые принцыпы}\label{section:base_principles}

\section{Базовые проблемы}\label{sections:base_problems}

% \section{Ещё примеры}





\section{Понятия из ML: признак, вектор признаков, выборка.}\label{section:ml_defs}



\end{document}